\documentclass[UTF8]{ctexart}

\begin{document}
 100,对于add在找到相同的key之后,判断值是否相同,相同返回false,不同继续遍历,直到curr.key>key;
 对于remove在找到相同的key之后,判断值是否相同,相同移除该项并返回,不同继续遍历,直到curr.key>key;
 对于contains判断链表中Node的值是否和想要的值相同,相同返回true,不同继续遍历,直到curr.key>key;

 118,我们称逻辑上被删除但物理上没有的x为x1,之后被添加的x为x2。
 在x1被逻辑删除后,有线程调用add(x2),插入x2,此时如果x2的插入位置在x1前面,则x1并没有被物理删除,
 但contains遍历数组最先找到并返回的是x2;如果x2插入的位置是x1的“后面”,则add中的find会在找到x1后直接进行
 物理删除,contains遍历数组也无法找到x1,因此这种情况不会发生。
\end{document}